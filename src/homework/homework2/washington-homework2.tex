%! Author = Len Washington III
%! Date = 9/18/25

% Preamble
\documentclass[
	number={2}
]{cs577homework}

% Document
\begin{document}

\maketitle

\problem{1}{5}
Draw the computational graph for the following function.
Then compute \texttt{wr.grad}, \texttt{wi.grad}, and \texttt{wo.grad} using backpropagation.

\begin{minted}{python}
x1 = ag.Scalar(2.0, label="z1\nleaf(x1)")
h0 = ag.Scalar(3.0, label="z2\nleaf(h0)")
wr = ag.Scalar(4.0, label="z3\nleaf(wr)")
wi = ag.Scalar(5.0, label="z4\nleaf(wi)")
wo = ag.Scalar(6.0, label="z5\nleaf(wo)")

z1 = x1
z2 = h0
z3 = wr
z4 = wi
z5 = z3 * z2 # wr * h0
z6 = z4 * z1 # wi * x1
z7 = z5 + z6
z8 = ag.relu(z7) # relu(wr * h0 + wi* x1)
z9 = wo
z10 = z8 * z9
z10.backward()

print(wr.grad, wi.grad, wo.grad)
\end{minted}

\noindent You are allowed to run the above using `\texttt{ex2.ipynb}' from Lecture 3 on Canvas.
However, you must explicitly explain step-by-step what happens during each iteration of the backpropagation, e.g., during the $0^{\text{th}}$ iteration, what is the node being visited in the computation graph.
For which nodes are the \texttt{grad} field updated?
Repeat this for the $1^{\text{st}}$, $2^{\text{nd}}$ and so on iterations.
A print-out of the computer-based calculation is not an acceptable answer.

\addanswer{1}
\problemline

\problem{2}{5}
Implement \texttt{def max(a, b)} for ag.Scalar
\begin{minted}{python}
def max(a: ag.Scalar, b: ag.Scalar):
	"""
	:param ag.Scalar a: input
	:param ag.Scalar b: input
	:return: Should be the maximum of a and b
	"""
	# [...]
	output._backward = _backward
	return output
\end{minted}
by filling in the function in \texttt{prob2.ipynb}.
Do this \texttt{def min(a, b)} as well.
For the backward function, if there are ties between \texttt{a.value} and \texttt{b.value} you can break ties arbitrarily.
There is a ``Grad check'' at the end of the Jupyter notebook.
If your implementation is correct, the Grad check code block should run silently.\\
\indent
Hint: if $f(a, b) = \max(a, b)$, what is $\frac{\partial}{\partial a}f(a, b)$ when $a \neq b$?

\problemline

\problem{3}{10}
Go to \texttt{prob3.ipynb} provided by filling in missing code block at ``\texttt{YOUR ANSWER HERE}''.

\end{document}